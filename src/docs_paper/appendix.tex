% Options for packages loaded elsewhere
\PassOptionsToPackage{unicode}{hyperref}
\PassOptionsToPackage{hyphens}{url}
%
\documentclass[
]{article}
\usepackage{amsmath,amssymb}
\usepackage{lmodern}
\usepackage{iftex}
\ifPDFTeX
  \usepackage[T1]{fontenc}
  \usepackage[utf8]{inputenc}
  \usepackage{textcomp} % provide euro and other symbols
\else % if luatex or xetex
  \usepackage{unicode-math}
  \defaultfontfeatures{Scale=MatchLowercase}
  \defaultfontfeatures[\rmfamily]{Ligatures=TeX,Scale=1}
\fi
% Use upquote if available, for straight quotes in verbatim environments
\IfFileExists{upquote.sty}{\usepackage{upquote}}{}
\IfFileExists{microtype.sty}{% use microtype if available
  \usepackage[]{microtype}
  \UseMicrotypeSet[protrusion]{basicmath} % disable protrusion for tt fonts
}{}
\makeatletter
\@ifundefined{KOMAClassName}{% if non-KOMA class
  \IfFileExists{parskip.sty}{%
    \usepackage{parskip}
  }{% else
    \setlength{\parindent}{0pt}
    \setlength{\parskip}{6pt plus 2pt minus 1pt}}
}{% if KOMA class
  \KOMAoptions{parskip=half}}
\makeatother
\usepackage{xcolor}
\IfFileExists{xurl.sty}{\usepackage{xurl}}{} % add URL line breaks if available
\IfFileExists{bookmark.sty}{\usepackage{bookmark}}{\usepackage{hyperref}}
\hypersetup{
  pdftitle={Appendix to ``Spatio-temporal estimates of HIV risk group proportions for adolescent girls and young women across 13 priority countries in sub-Saharan Africa''},
  hidelinks,
  pdfcreator={LaTeX via pandoc}}
\urlstyle{same} % disable monospaced font for URLs
\usepackage[margin=1in]{geometry}
\usepackage{graphicx}
\makeatletter
\def\maxwidth{\ifdim\Gin@nat@width>\linewidth\linewidth\else\Gin@nat@width\fi}
\def\maxheight{\ifdim\Gin@nat@height>\textheight\textheight\else\Gin@nat@height\fi}
\makeatother
% Scale images if necessary, so that they will not overflow the page
% margins by default, and it is still possible to overwrite the defaults
% using explicit options in \includegraphics[width, height, ...]{}
\setkeys{Gin}{width=\maxwidth,height=\maxheight,keepaspectratio}
% Set default figure placement to htbp
\makeatletter
\def\fps@figure{htbp}
\makeatother
\setlength{\emergencystretch}{3em} % prevent overfull lines
\providecommand{\tightlist}{%
  \setlength{\itemsep}{0pt}\setlength{\parskip}{0pt}}
\setcounter{secnumdepth}{-\maxdimen} % remove section numbering
\newlength{\cslhangindent}
\setlength{\cslhangindent}{1.5em}
\newlength{\csllabelwidth}
\setlength{\csllabelwidth}{3em}
\newlength{\cslentryspacingunit} % times entry-spacing
\setlength{\cslentryspacingunit}{\parskip}
\newenvironment{CSLReferences}[2] % #1 hanging-ident, #2 entry spacing
 {% don't indent paragraphs
  \setlength{\parindent}{0pt}
  % turn on hanging indent if param 1 is 1
  \ifodd #1
  \let\oldpar\par
  \def\par{\hangindent=\cslhangindent\oldpar}
  \fi
  % set entry spacing
  \setlength{\parskip}{#2\cslentryspacingunit}
 }%
 {}
\usepackage{calc}
\newcommand{\CSLBlock}[1]{#1\hfill\break}
\newcommand{\CSLLeftMargin}[1]{\parbox[t]{\csllabelwidth}{#1}}
\newcommand{\CSLRightInline}[1]{\parbox[t]{\linewidth - \csllabelwidth}{#1}\break}
\newcommand{\CSLIndent}[1]{\hspace{\cslhangindent}#1}
\RequirePackage{amsthm,amsmath,amsfonts,amssymb}
\RequirePackage[authoryear]{natbib}
\RequirePackage{bm}
\RequirePackage{enumerate}
\RequirePackage{enumitem}
\RequirePackage{tabularx}
\RequirePackage{adjustbox}
\RequirePackage{tikz}
\RequirePackage{bbm}
\RequirePackage{lscape}
\RequirePackage{booktabs}
\RequirePackage{longtable}
\RequirePackage{lscape}
\RequirePackage{pifont}
\usepackage{footnote}

\newcommand{\Sc}{\mathcal{S}}
\newcommand{\R}{\mathcal{R}}
\newcommand{\N}{\mathcal{N}}
\newcommand{\X}{\mathcal{X}}
\newcommand{\m}{\mathbf{m}}
\newcommand{\bu}{\mathbf{u}}
\newcommand{\bv}{\mathbf{v}}
\newcommand{\w}{\mathbf{w}}
\newcommand{\x}{\mathbf{x}}
\newcommand{\y}{\mathbf{y}}
\newcommand{\z}{\mathbf{z}}
\newcommand{\bb}{\mathbf{b}}
\newcommand{\brho}{\bm{\rho}}
\newcommand{\bphi}{\bm{\phi}}
\newcommand{\btheta}{\bm{\theta}}
\newcommand{\bpsi}{\bm{\psi}}
\ifLuaTeX
  \usepackage{selnolig}  % disable illegal ligatures
\fi

\title{Appendix to ``Spatio-temporal estimates of HIV risk group
proportions for adolescent girls and young women across 13 priority
countries in sub-Saharan Africa''}
\author{}
\date{\vspace{-2.5em}}

\begin{document}
\maketitle

\newpage

\hypertarget{statistical-modelling-of-risk-group-proportions}{%
\section{Statistical modelling of risk group
proportions}\label{statistical-modelling-of-risk-group-proportions}}

\hypertarget{overview}{%
\subsection{Overview}\label{overview}}

We index the four risk groups given in Table by \(k \in \{1, 2, 3, 4\}\)
and we denote being in either the third or fourth risk group by
\(k = 3^{+}\). Let \(c \in \{1, \ldots, 13\}\) denote the 13 studied
AGYW priority countries, each partitioned into districts. For each
district \(i \in \{1, \ldots, n\}\), year
\(t \in \{1999, \ldots, 2018\}\), age category
\(a \in \{\text{15--19}, \text{20--24}, \text{25--29}\}\), and risk
group \(k \in \{1, 2, 3^{+}\}\) we consider a multinomial random vector
\begin{equation}
    \y_{ita} = (y_{ita1}, \ldots, y_{ita3^{+}})^\top \sim \text{Multinomial}(m_{ita}; \, p_{ita1}, \ldots, p_{ita3^{+}}).
\end{equation} The number of women in risk group \(k\) is \(y_{itak}\)
and the sample size is \(m_{ita} = \sum_{k = 1}^{3^{+}} y_{itak}\),
where \(p_{itak} > 0\) is the probability of membership of the \(k\)th
risk group with \(\sum_{k = 1}^{3^{+}} p_{itak} = 1\). To efficiently
model this data we used the multinomial-Poisson transformation, which
allows multinomial logistic regression models to be cast as equivalent
Poisson log-linear models (Baker 1994; Lee, Green, and Ryan 2017)
\begin{align}
    y_{itak} &\sim \text{Poisson}(\kappa_{itak}), \label{eq:poisson} \\
    \log(\kappa_{itak}) &= \eta_{itak}, \label{eq:linearpredictor}
\end{align} for particular choice of the linear predictor
\(\eta_{itak}\), described in more detail in Section.

To estimate the proportion of those in the \(k = 3^{+}\) risk group that
were respectively in the \(k = 3\) and \(k = 4\) risk groups, we fit a
separate logistic regression model of the form \begin{align}
    y_{ia4} &\sim \text{Binomial} \left( y_{ia3} + y_{ia4}, q_{ia} \right), \\
    q_{ia} &= p_{ia4} / (p_{ia3} + p_{ia4}) = p_{ia4} / p_{ia{3^+}}, \\
    q_{ia} &= \text{logit}^{-1} \left( \eta_{ia} \right), 
\end{align} for choice of the linear predictor \(\eta_{ia}\), described
in more detail in Section. This approach allowed us to include all
surveys in the multinomial regression model, but only those surveys with
a specific transactional sex question to estimate the proportions
\(\{p_{ia4}\}\). As all such surveys occurred in 2013-2018, \(q_{ia}\)
was assumed to be constant over time.

To quantify uncertainty from our model, we took posterior samples
\(s = 1, \ldots, 1000\) from each of the multinomial \(\{p_{itak}^s\}\)
and logistic regression \(\{q_{ia}^s\}\) models. Samples \(p_{ita3}^s\)
and \(p_{ita4}^s\) were then generated by splitting samples from the
\(k = 3^{+}\) category of the multinomial model via \begin{align*}
    p_{ita3}^s &= (1 - q_{ia}^s)p_{ita{3^+}^s}, \\
    p_{ita4}^s &= q_{ia}^s p_{ita{3^+}}^s.
\end{align*} As transactional sex is too inclusive to directly
correspond to sex work (Wamoyi et al. 2016), we adjusted our samples so
that the population size estimates for FSW aggregated to a national
level matched those of Stevens et al. (2022). In making this adjustment,
we assumed that subnational variation in the FSW proportion corresponds
to that of the transactional sex proportion. We calculated the mean,
median, 2.5\% and 97.5\% quantiles for the risk group probabilities
empirically using these adjusted samples. To produce aggregated
estimates, such as the age category 15-24 and national estimates, we
weighted the adjusted samples by population sizes
\(N_{ita} = \sum_k N_{itak}\) obtained from the Naomi model (Eaton et
al. 2021).

\hypertarget{multinomial-regression-model}{%
\subsection{Multinomial regression
model}\label{multinomial-regression-model}}

\hypertarget{the-multinomial-poisson-transformation}{%
\subsubsection{The multinomial-Poisson
transformation}\label{the-multinomial-poisson-transformation}}

The multinomial-Poisson transformation reframes a given multinomial
logisitic regression model as an equivalent Poisson log-linear model of
the form \begin{align}
    y_{itak} &\sim \text{Poisson}(\kappa_{itak}), \\
    \log(\kappa_{itak}) &= \eta_{itak},
\end{align} for suitable choice of linear predictor \(\eta_{itak}\).
Category probabilities are then obtained by applying the softmax
function to the linear predictor as follows \begin{equation}
    p_{itak} = \frac{\exp(\eta_{itak})}{\sum_{k = 1}^{3^{+}} \exp(\eta_{itak})} = \frac{\kappa_{itak}}{\sum_{k = 1}^{3^{+}} \kappa_{itak}}.
\end{equation} In the equivalent model, the sample sizes
\(m_{ita} = \sum_k y_{itak}\) are treated as random rather than fixed as
they would be in the multinomial logistic regression model. In
particular, following Equation, the sample sizes follow a Poisson
distribution \begin{equation}
    m_{ita} \sim \text{Poisson} \left( \kappa_{ita} \right),
\end{equation} where the intensity
\(\kappa_{ita} = \sum_k \kappa_{itak}\) is given by a sum of intensities
over each category. The basis of the Poisson trick is that, conditional
on their sum, Poisson counts are jointly multinomially distributed
(McCullagh and Nelder 1989) \begin{equation}
    \mathbf{y}_{ita} \, | \, m_{ita} \sim \text{Multinomial} \left( m_{ita}; \frac{\kappa_{ita1}}{\kappa_{ita}}, \ldots, \frac{\kappa_{itaK}}{\kappa_{ita}} \right).
\end{equation} As such, the joint distribution of
\(p(\mathbf{y}_{ita}, m_{ita})\) is \begin{align}
p(m_{ita}) p(\mathbf{y}_{ita} \, | \, m_{ita}) 
&= \exp(-\kappa_{ita}) \frac{(\kappa_{ita})^{m_{ita}}}{m_{ita}!} \times \frac{m_{ita}!}{\prod_k y_{itak}!} \prod_k \left( \frac{\kappa_{itak}}{\kappa_{ita}} \right)^{y_{itak}} \\
&= \prod_k \left( \frac{\exp(-\kappa_{itak}) \left( \kappa_{itak} \right)^{y_{itak}}}{y_{itak}!} \right) \\
&= \prod_k \text{Poisson} \left( y_{itak} \, | \, \kappa_{itak} \right). \label{eq:prodpoisson}
\end{align} Equation corresponds to the product of \(K\) independent
Poisson likelihoods, and the joint distribution over all the data
\(\{\mathbf{y}_{ita}, m_{ita}\}_{i, t, a}\) is given by
\(\prod_{i, t, a, k} \text{Poisson} \left( y_{itak} \, | \, \kappa_{itak} \right)\).
The Poisson log-linear model, which vicariously includes the sample
sizes as random, is equivalent only when these normalisation constants
are recovered exactly by the model. To ensure that this is the case, we
included observation-specific random effects \(\theta_{ita}\) in the
equation for the linear predictor in every model we considered.
Multiplying each of \(\{\kappa_{itak}\}_{k = 1}^K\) by
\(\exp(\theta_{ita})\) has no effect on the multinomial probabilities,
but does provide the flexibility required for \(\kappa_{ita}\) to
exactly recover \(m_{ita}\) as desired. Although in theory an improper
prior \(\theta_{ita} \propto 1\) should be used, in practise, by keeping
\(\eta_{ita}\) small so that arbitrarily large values of the
observation-specific random effects \(\theta_{ita}\) are not required,
it is sufficient (and practically preferable for inference) to instead
use a vague prior. As such, we set
\(\theta_{ita} \sim \mathcal{N}(0, \tau_\theta^{-1})\), with fixed
precision \(\tau_\theta = 0.000001\) (or a standard deviation of
\(\sigma_\theta = 1000\)).

\hypertarget{model-specifications}{%
\subsubsection{Model specifications}\label{model-specifications}}

We considered eight models
(Table\textasciitilde{}\ref{tab:multinomial-models}) for \(\eta_{ita}\)
of the form
\(\eta_{ita} = \theta_{ita} + \beta_k + \alpha_{ak} + \zeta_{ck} + \phi_{ik} + \gamma_{tk} + \delta_{itk}\).
Observation random effects \(\theta_{ita}\) with a diffuse prior,
\(\mathcal{N}(0, 1000^2)\), were included in every model to ensure the
sample sizes \(m_{ita}\) were recovered exactly, as required for the
Poisson log-linear model to be equivalent to the multinomial logistic
regression. To capture age-specific proportion estimates for each
category \(k = 1, 2, 3^{+}\), we included category random effects
\(\beta_k \sim \mathcal{N}(0, \tau_\beta^{-1})\) and age-category random
effects \(\alpha_{ak} \sim \mathcal{N}(0, \tau_\alpha^{-1})\). Owing to
country-level heterogeneity in the data, we also included
country-category random effects
\(\zeta_{ck} \sim \mathcal{N}(0, \tau_\zeta^{-1})\). We considered two
specifications, independent and identically distributed (IID) and Besag,
for the space-category \(\phi_{ik}\) random effects and two
specifications, IID and first order autoregressive (AR1), for the
year-category \(\gamma_{tk}\) random effects, as well as separable
space-year-category interactions \(\delta_{itk}\) with corresponding
Kronecker product structure. All random effect precision parameters
\(\tau \in \{\tau_\beta, \tau_\alpha, \tau_\zeta, \tau_\phi, \tau_\gamma\, \tau_\delta\}\)
were given independent penalised complexity (PC) priors (Simpson et al.
2017) with base model \(\sigma = 0\) given by
\(p(\tau) = 0.5 \nu \tau^{-3/2} \exp \left( - \nu \tau^{-1/2} \right)\)
where \(\nu = - \ln(0.01) / 2.5\) such that
\(\mathbb{P}(\sigma > 2.5) = 0.01\).

To ensure interpretable posterior inferences of random effect
contribution, we applied sum-to-zero constraints such that the
age-category, country-category, space-category and year-category random
effects did not alter overall category probabilities. For the
space-year-category random effects, we analogously applied sum-to-zero
constraints to maintain roles of the space-category and year-category
random effects. Together, these
were\footnote{The constraints for the space-year-category random effects were unable to be implemented fully, though this only effects the variance components interpretation.}:

\begin{enumerate}
    \item Category $\sum_k \beta_k = 0$
    \item Age $\sum_a \alpha_{ak} = 0, \, \forall \, k$
    \item Country $\sum_c \zeta_{ck} = 0, \, \forall \, k$
    \item Spatial $\sum_i \phi_{ik} = 0, \, \forall \, k$
    \item Temporal $\sum_t \gamma_{tk} = 0, \, \forall \, k$
    \item Spatio-temporal $\sum_i \delta_{itk} = 0, \, \forall \, t, k;
\quad \sum_t \delta_{itk} = 0, \, \forall \, i, k;
\quad \sum_k \delta_{itk} = 0, \, \forall \, i, t$
\end{enumerate}

\hypertarget{spatial-random-effects}{%
\paragraph{\texorpdfstring{Spatial random effects
\label{sec:spatial-random-effects}}{Spatial random effects }}\label{spatial-random-effects}}

The specifications we considered for the space-category random effects
were IID \(\phi_{ik} \sim \mathcal{N}(0, \tau_\phi^{-1})\) and Besag
grouped by category \[
\bphi = (\phi_{11}, \ldots, \phi_{n1}, \ldots, \phi_{1K}, \ldots \phi_{nK})^\top \sim \mathcal{N}(\mathbf{0}, (\tau_\phi \mathbf{R}^\star_\phi)^{-})
\] where the scaled structure matrix
\(\mathbf{R}^\star_\phi = \mathbf{R}^\star_b \otimes \mathbf{I}\) is
given by the Kronecker product of the scaled Besag structure matrix
\(\mathbf{R}^\star_b\) and the identity matrix \(\mathbf{I}\). Scaling
of the structure matrix to have generalised variance one ensures
interpretable priors may be placed on the precision parameter (Sørbye
and Rue 2014). We follow the further recommendations of
Freni-Sterrantino, Ventrucci, and Rue (2018) with regard to disconnected
adjacency graphs, singletons and constraints. The Besag structure matrix
\(\mathbf{R}_b\) is obtained by the precision matrix of the random
effects \(\mathbf{b} = (b_1, \ldots, b_n)^\top\) with the full
conditionals \begin{equation}
b_i \, | \, \mathbf{b}_{-i} \sim \N\left(\frac{\sum_{j: j \sim i} b_j}{n_{\delta i}}, \frac{1}{n_{\delta i}}\right),
\end{equation} where \(j \sim i\) if the districts \(A_i\) and \(A_j\)
are adjacent, and \(n_{\delta i}\) is the number of districts adjacent
to \(A_i\).

In preliminary testing, we tried excluding spatial random effects from
the model, but found that this negatively effected performance. We also
tried using the BYM2 model (Simpson et al. 2017) in place of the Besag,
but found that the proportion parameter posteriors tended to be highly
peaked at the value one. For simplicity and to avoid numerical issues,
by instead using Besag random effects we decided to fix this proportion
to one, rather than learn it.

\hypertarget{temporal-random-effects}{%
\paragraph{\texorpdfstring{Temporal random effects
\label{sec:temporal-random-effects}}{Temporal random effects }}\label{temporal-random-effects}}

The specifications we considered for the survey-category random effects
were IID \(\phi_{tk} \sim \mathcal{N}(0, \tau_\phi^{-1})\) and AR1
grouped by category \[
\bm{\gamma} = (\gamma_{11}, \ldots, \gamma_{1K}, \ldots, \gamma_{T1}, \ldots, \gamma_{TK})^\top \sim \mathcal{N}(\mathbf{0}, (\tau_\phi \mathbf{R}^\star_\gamma)^{-})
\] where the scaled structure matrix
\(\mathbf{R}^\star_\gamma = \mathbf{R}^\star_r \otimes \mathbf{I}\) is
given by the Kronecker product of a scaled AR1 structure matrix
\(\mathbf{R}^\star_r\) and the identity matrix \(\mathbf{I}\). The AR1
structure matrix \(\mathbf{R}_r\) is obtained by precision matrix of the
random effects \(\mathbf{r} = (r_1, \ldots, r_T)^\top\) specified by
\begin{align}
r_1 &\sim \left( 0, \frac{1}{1 - \rho^2} \right), \\
r_t &= \rho r_{t - 1} + \epsilon_t, \quad t = 2, \ldots, T, 
\end{align} where \(\epsilon_t \sim \mathcal{N}(0, 1)\) and
\(|\rho| < 1\). For the lag-one correlation parameter \(\rho\), we used
the PC prior, as derived by Sørbye and Rue (2017), with base model
\(\rho = 1\) and condition \(\mathbb{P}(\rho > 0 = 0.75)\). We chose a
base model corresponding to no change in behaviour over time, rather
than the alternative \(\rho = 0\) corresponding to no correlation in
behaviour over time, as we judged the former to be more plausible a
priori.

\hypertarget{spatio-temporal-random-effects}{%
\paragraph{\texorpdfstring{Spatio-temporal random effects
\label{sec:spatio-temporal-random-effects}}{Spatio-temporal random effects }}\label{spatio-temporal-random-effects}}

For each model including spatio-temporal random effects we used the
specification \begin{equation}
    \bm{\delta} = (\delta_{111}, \ldots, \delta_{nTK})^\top \sim \mathcal{N}(\mathbf{0}, (\tau_\delta \mathbf{R}^\star_\delta)^{-}),
\end{equation} where \(\mathbf{R}^\star_\delta\) is a Kronecker product
of the relevant space, time and category structure matrices. In
particular, for each of the four models including spatio-temporal
interactions, these matrices are

\begin{itemize}
\item
1x: No spatial or temporal structure (Type I) $\mathbf{R}^\star_\delta = \mathbf{I} \otimes \mathbf{I} \otimes \mathbf{I}$,
\item
2x: Structured spatial and unstructured temporal (Type II) $\mathbf{R}^\star_\delta = \mathbf{R}^\star_b \otimes \mathbf{I} \otimes \mathbf{I}$,
\item
3x: Unstructured spatial and structured temporal (Type III) $\mathbf{R}^\star_\delta = \mathbf{I} \otimes \mathbf{R}^\star_a \otimes \mathbf{I}$,
\item
4x: Structured spatial and temporal (Type IV) $\mathbf{R}^\star_\delta = \mathbf{R}^\star_b \otimes \mathbf{R}^\star_a \otimes \mathbf{I}$,
\\
\end{itemize}

where the first, second and third elements of the product represent
space, time and category respectively, and the interaction type provided
in brackets is given according to the Knorr-Held (2000) framework.

\hypertarget{survey-weighted-likelihood}{%
\subsubsection{Survey weighted
likelihood}\label{survey-weighted-likelihood}}

The surveys we used utilise a complex design, in which each individual
has an unequal probability of being included in the sample. For example,
the DHS often employs a two-stage cluster design, first taking an urban
rural stratified sample of ennumeration areas, before selecting
households from each ennumeration area using systematic sampling. To
account for the survey design, we use a weighted pseudo-likelihood where
the observed counts \(y_{itak}\) are replaced by effective counts
\(y^\star_{itak}\) calculated using survey weights \(w_{itak}\). We
multiplied direct estimates produced using the \texttt{survey} package
\citep{JSSv009i08} by the Kish effective sample size
\citep{kish1965survey} \begin{equation}
    m^\star_{itak} = \left(\sum_{itak} w_{itak} \right)^2 / \sum_{itak} {w_{itak}}^2
\end{equation} to obtain \(y^\star_{itak}\). As these counts may not be
integers, the Poisson likelihood in
Equation\textasciitilde{}\eqref{eq:poisson} is inappropriate. Instead,
we used a generalised Poisson pseudo-likelihood
\(y^\star \sim \text{xPoisson}(\kappa)\), as implemented by
\texttt{family\ =\ "xPoisson"} in \texttt{R-INLA}, given by
\begin{equation}
    p(y^\star) = \frac{\kappa^{y^\star}}{\left \lfloor{y^\star!}\right \rfloor } \exp \left(- \kappa \right).
\end{equation}

\newpage

\hypertarget{prevalence-incidence-and-new-infections-reached-analysis}{%
\section{\texorpdfstring{Prevalence, incidence and new infections
reached analysis
\label{app:inc-inf}}{Prevalence, incidence and new infections reached analysis }}\label{prevalence-incidence-and-new-infections-reached-analysis}}

\hypertarget{calculation-of-prevalence-and-plhiv}{%
\subsection{Calculation of prevalence and
PLHIV}\label{calculation-of-prevalence-and-plhiv}}

We calculated HIV prevalence \(\rho_{iak}\) and number of people living
with HIV (PLHIV) \(H_{iak} = \rho_{iak} N_{iak}\) stratified according
to district, age group and risk group by disaggregation \begin{align}
    H_{ia} &= \sum_k H_{iak} = \sum_k \rho_{iak} N_{iak} \\
    &= \rho_{ia1} N_{ia1} + \rho_{ia1} \text{PR}_2 N_{ia2} + \rho_{ia1} \text{PR}_3 N_{ia3} + \rho_{ia1} \text{PR}_4 N_{ia4} \\
    &= \rho_{ia1} \left(N_{ia1} + \text{PR}_2 N_{ia2} + \text{PR}_3 N_{ia3} + \text{PR}_4 N_{ia4} \right).
\end{align} Risk group specific prevalence estimates are then given by
\begin{align}
    \rho_{ia1} &= H_{ia} / \left(N_{ia1} + \text{PR}_2 N_{ia2} + \text{PR}_3 N_{ia3} + \text{PR}_4 N_{ia4} \right), \\
    \rho_{ia2} &= \rho_{ia1} \text{PR}_2,\\
    \rho_{ia3} &= \rho_{ia1} \text{PR}_3,\\
    \rho_{ia4} &= \rho_{ia1} \text{PR}_4,
\end{align} which we evaluated using Naomi model estimates of PLHIV
\(H_{ia} = \rho_{ia} N_{ia}\), prevalence ratios \(\{\text{PR}_k\}\) as
described below, and risk group population sizes
\(N_{iak} = p_{iak} N_{ia}\) based on our 2018 risk group estimates
together with population size estimates from the Naomi model. To obtain
the prevalence ratios, we\ldots{}

\hypertarget{calculation-of-incidence-and-number-of-new-infections}{%
\subsection{Calculation of incidence and number of new
infections}\label{calculation-of-incidence-and-number-of-new-infections}}

Similarly, we calculated HIV incidence \(\lambda_{iak}\) and number of
new HIV infections \(I_{iak}\) stratified according to district, age
group and risk group by disaggregation \begin{align}
    I_{ia} &= \sum_k I_{iak} = \sum_k \lambda_{iak}N_{iak} \\
    &= 0 + \lambda_{ia2} N_{ia2} + \lambda_{ia3} N_{ia3} + \lambda_{ia4} N_{ia4} \\
    &= \lambda_{ia2} \left(N_{ia2}  + \text{RR}_{3} N_{ia3} + \text{RR}_4(\lambda_{ia}) N_{ia4}  \right).
\end{align} Risk group specific incidence estimates are then given by
\begin{align}
    \lambda_{ia1} &= 0, \\
    \lambda_{ia2} &= I_{ia} / \left(N_{ia2} + \text{RR}_{3} N_{ia3} + \text{RR}_4(\lambda_{ia}) N_{ia4}\right), \\
    \lambda_{ia3} &= \text{RR}_{3} \lambda_{ia2}, \\
    \lambda_{ia4} &= \text{RR}_4(\lambda_{ia}) \lambda_{ia2}.
\end{align} which we evaluated using Naomi model estimates of the number
of new HIV infections \(I_{ia} = \lambda_{ia} N_{ia}\), HIV infection
risk ratios \(\{\text{RR}_3, \text{RR}_4(\lambda_{ia})\}\) as given in
Table\textasciitilde{}\ref{tab:risk-categories}, and 2018 risk group
population sizes as above. The risk ratio \(\text{RR}_4(\lambda_{ia})\)
is defined as a function of general population incidence. Using these
estimates \(\{\lambda_{iak}\}\), the number of new HIV infections is
then \(I_{iak} = \lambda_{iak} N_{iak}\).

\hypertarget{analysis-of-new-infections-reached}{%
\subsection{Analysis of new infections
reached}\label{analysis-of-new-infections-reached}}

We calculated the number of new infections that would be reached
prioritising according to each possible stratification of the
population. As an illustration, for stratification by age, we aggregated
the number of new HIV infections and HIV incidence as such \begin{align}
    I_a &= \sum_{ik} I_{iak}, \\
    \lambda_a &= I_a / \sum_{ik} N_{iak}.
\end{align} Under this stratification, individuals in each age group
\(a\) would be prioritised according to the highest HIV incidence
\(\lambda_a\). We repeated this analysis for all \(2^3 = 8\) possible
combinations of stratification by location, age, and risk group.

\newpage

\hypertarget{the-global-aids-strategy-2021-2016-as-it-relates-to-agyw}{%
\section{The Global AIDS Strategy 2021-2016 as it relates to
AGYW}\label{the-global-aids-strategy-2021-2016-as-it-relates-to-agyw}}

\begin{footnotesize}

\begin{table}[h]
\centering
\begin{tabularx}{\textwidth}{lX}
\toprule
Prioritisation strata & Criterion \\ 
 \midrule
Low & 0.3-1\% incidence AND low-risk behaviour OR $<$0.3\% incidence AND high-risk behaviour \\
Moderate & 1-3\% incidence AND low-risk behaviour OR 0.3-1\% incidence AND high-risk behaviour \\
High & 1-3\% incidence AND high-risk behaviour \\
Very high & $>$3\% incidence \\
\bottomrule
\end{tabularx}
\caption{Prioritisation strata.}
\label{tab:unaids-strategy-prioritisation}
\end{table}

\end{footnotesize}

\begin{footnotesize}

\begin{table}[h]
\centering
\begin{tabularx}{\textwidth}{lX}
\toprule
Intervention & Details \\ 
 \midrule
STI screening and treatment & (Remain to be filled in: describe targets and commitments in terms of each prioritisation strata) \\
PrEP use & (As above) \\
Access to PEP & (As above) \\
Comprehensive sexuality education & (As above) \\
Economic empowerment & (As above) \\
\bottomrule
\end{tabularx}
\caption{Targets and commitments.}
\label{tab:unaids-strategy-targets}
\end{table}

\end{footnotesize}

\newpage

\hypertarget{survey-questions-and-risk-group-classification}{%
\section{Survey questions and risk group
classification}\label{survey-questions-and-risk-group-classification}}

\hypertarget{survey-questions}{%
\subsection{Survey questions}\label{survey-questions}}

\begin{footnotesize}

\begin{table}[h]
\centering
\begin{tabularx}{\textwidth}{lX}
\toprule
Variable(s) & Description \\ 
 \midrule
\texttt{v501} & Current marital status of the respondent. \\
\texttt{v529} & Computed time since last sexual intercourse. \\
\texttt{v531} & Age at first sexual intercourse--imputed. \\
\texttt{v766b} & Number of sexual partners during the last 12 months (including husband). \\
\texttt{v767[a, b, c]} & Relationship with last three sexual partners. Options are: spouse, boyfriend not living with respondent, other friend, casual acquaintance, relative, commercial sex worker, live-in partner, other. \\
\texttt{v791a} & Had sex in return for gifts, cash or anything else in the past 12 months. Asked only to women 15-24 who are not in a union. \\
\bottomrule
\end{tabularx}
\caption{AIS, BAIS and DHS survey questions.}
\end{table}

\end{footnotesize}

\begin{footnotesize}

\begin{table}[h]
\centering
\begin{tabularx}{\textwidth}{lX}
\toprule
Variable(s) & Description \\ 
 \midrule
\texttt{part12monum} & Number of sexual partners during the last 12 months (including husband). \\
\texttt{part12modkr} & Reason for leaving \texttt{part12monum} blank. \\
\texttt{partlivew[1, 2, 3]} & Does the person you had sex with live in this household? \\
\texttt{partrelation[1, 2, 3]} & Relationship with last three sexual partners. Options are: husband, live-in partner, partner (not living with), ex-spouse/partner, friend/acquaintance, sex worker, sex worker client, stranger, other, don't know, refused. \\
\texttt{sellsx12mo} & Had sex for money and/or gifts in the last 12 months. \\
\texttt{buysx12mo} & Paid money or given gifts for sex in the last 12 months. \\
\bottomrule
\end{tabularx}
\caption{PHIA survey questions.}
\end{table}

\end{footnotesize}

\newpage

\subsection{Risk group classification based on survey questions}

\begin{figure}[h!]
    \centering
    \includegraphics{depends/category-flowchart.pdf}
    \caption{Flowchart describing allocation of respondents to risk groups.}
    \label{fig:category-flowchart}
\end{figure}

\newpage

\hypertarget{supplementary-figures}{%
\section{Supplementary figures}\label{supplementary-figures}}

\newpage

\hypertarget{supplementary-tables}{%
\section{Supplementary tables}\label{supplementary-tables}}

\hypertarget{references}{%
\section*{References}\label{references}}
\addcontentsline{toc}{section}{References}

\hypertarget{refs}{}
\begin{CSLReferences}{1}{0}
\leavevmode\vadjust pre{\hypertarget{ref-baker1994multinomial}{}}%
Baker, Stuart G. 1994. {``{The multinomial-Poisson transformation}.''}
\emph{Journal of the Royal Statistical Society: Series D (The
Statistician)} 43 (4): 495--504.

\leavevmode\vadjust pre{\hypertarget{ref-eaton2021naomi}{}}%
Eaton, Jeffrey W., Laura Dwyer-Lindgren, Steve Gutreuter, Megan
O'Driscoll, Oliver Stevens, Sumali Bajaj, Rob Ashton, et al. 2021.
{``{Naomi: a new modelling tool for estimating HIV epidemic indicators
at the district level in sub-Saharan Africa}.''} \emph{Journal of the
International AIDS Society} 24 (S5): e25788.

\leavevmode\vadjust pre{\hypertarget{ref-freni2018note}{}}%
Freni-Sterrantino, Anna, Massimo Ventrucci, and Håvard Rue. 2018. {``A
Note on Intrinsic Conditional Autoregressive Models for Disconnected
Graphs.''} \emph{Spatial and Spatio-Temporal Epidemiology} 26: 25--34.

\leavevmode\vadjust pre{\hypertarget{ref-knorr2000bayesian}{}}%
Knorr-Held, Leonhard. 2000. {``Bayesian Modelling of Inseparable
Space-Time Variation in Disease Risk.''} \emph{Statistics in Medicine}
19 (17-18): 2555--67.

\leavevmode\vadjust pre{\hypertarget{ref-lee2017poisson}{}}%
Lee, Jarod YL, Peter J Green, and Louise M Ryan. 2017. {``{On the
{`Poisson Trick'} and its Extensions for Fitting Multinomial Regression
Models}.''} \emph{arXiv Preprint arXiv:1707.08538}.

\leavevmode\vadjust pre{\hypertarget{ref-mccullagh1989generalized}{}}%
McCullagh, Peter, and John A Nelder. 1989. \emph{Generalized Linear
Models}. Routledge.

\leavevmode\vadjust pre{\hypertarget{ref-simpson2017penalising}{}}%
Simpson, Daniel, Håvard Rue, Andrea Riebler, Thiago G Martins, and
Sigrunn H Sørbye. 2017. {``Penalising Model Component Complexity: A
Principled, Practical Approach to Constructing Priors.''}
\emph{Statistical Science} 32 (1): 1--28.

\leavevmode\vadjust pre{\hypertarget{ref-sorbye2014scaling}{}}%
Sørbye, Sigrunn Holbek, and Håvard Rue. 2014. {``{Scaling intrinsic
Gaussian Markov random field priors in spatial modelling}.''}
\emph{Spatial Statistics} 8: 39--51.

\leavevmode\vadjust pre{\hypertarget{ref-sorbye2017penalised}{}}%
---------. 2017. {``Penalised Complexity Priors for Stationary
Autoregressive Processes.''} \emph{Journal of Time Series Analysis} 38
(6): 923--35.

\leavevmode\vadjust pre{\hypertarget{ref-stevens2022estimating}{}}%
Stevens, Oliver, Keith Sabin, Sonia Arias Garcia, Kalai Willis, Abu
Abdul-Quader, Anne McIntyre, Frances Cowan, et al. 2022. {``{Estimating
key population size, HIV prevalence, and ART coverage for sub-Saharan
Africa at the national level}.''}

\leavevmode\vadjust pre{\hypertarget{ref-wamoyi2016transactional}{}}%
Wamoyi, Joyce, Kirsten Stobeanau, Natalia Bobrova, Tanya Abramsky, and
Charlotte Watts. 2016. {``{Transactional sex and risk for HIV infection
in sub-Saharan Africa: a systematic review and meta-analysis}.''}
\emph{Journal of the International AIDS Society} 19 (1): 20992.

\end{CSLReferences}

\end{document}
