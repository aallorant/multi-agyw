\documentclass[tikz]{standalone}

\usepackage{tikz}
\usetikzlibrary{shapes,arrows,positioning}

\begin{document}

\tikzstyle{question} = [rectangle, draw, text width=10em, text centered, rounded corners, minimum height=4em]
\tikzstyle{category} = [rectangle, draw, text width=10em, text centered, rounded corners, minimum height=4em]
\tikzstyle{outerblock} = [rectangle, draw, text width=11em, text centered, rounded corners, minimum height=22em]
\tikzstyle{line} = [draw, -latex']

\begin{tikzpicture}[node distance = 2cm, auto]
    \node[category] at (0, 0) (1) {$k = 1$:\\Not sexually active};
    \node[category] at (0, -2) (2) {$k = 4$:\\Female sex worker};
    \node[category] at (0, -4) (3) {$k = 2$:\\One cohabiting sexual partner};
    \node[category] at (0, -6) (4) {$k = 3$:\\Non-regular sexual partner(s)};
    % \node[outerblock] at (0, -3) {};

    \node[question] at (-10, 0) (a) {Is the respondent sexually active in the past 12 months?};
    \node[question] at (-10, -2) (b) {Has the respondent been given gifts or money for sex?};
    \node[question] at (-5, -4) (c) {Does that partner live in the same household as the respondent?};
    \node[question] at (-10, -4) (d) {Number of partners of the respondent of the past 12 months?};

    \path[line] (a) -- node {No}(1);
    \path[line] (a) -- node {Yes}(b);
    \path[line] (b) -- node {Yes}(2);
    \path[line] (b) -- node {No}(d);
    \path[line] (c) -- node {Yes}(3);
    \path[line] (d) -- node {$1$}(c);
    \path[line] (d) |- node[near start] {$>1$}(4);
    \path[line] (c) |- node[near start] {No}(4);
\end{tikzpicture}

\end{document}
